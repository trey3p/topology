\documentclass{article}
\author{Trey Plante}


\title{Primer on Topology for AlgTop \\
  \normalsize This is a collection of notes intended to provide the minimal necessary topology background to ramp someone up to algebraic topology.}
\date{}
\usepackage[normalem]{ulem}
\usepackage{ebproof}
\usepackage{amsmath,amsthm,amssymb,stmaryrd}
\usepackage[alphabetic]{amsrefs}
\usepackage{hyperref, cleveref}
\usepackage[inline]{enumitem}
\usepackage{mathtools}

%===========================================================
\usepackage{amsmath,amsthm,amssymb,stmaryrd}
\usepackage[alphabetic]{amsrefs}
\usepackage{hyperref, cleveref}
\usepackage[inline]{enumitem}
\usepackage{lipsum}

\theoremstyle{plain}
\newtheorem{theorem}{Theorem}[section]
\crefname{theorem}{Theorem}{Theorems}

\newtheorem{lemma}[theorem]{Lemma}
\crefname{lemma}{Lemma}{Lemmas}

\newtheorem{proposition}[theorem]{Proposition}
\crefname{proposition}{Proposition}{Propositions}

\newtheorem{corollary}[theorem]{Corollary}
\crefname{corollary}{Corollary}{Corollaries}

\newtheorem{conjecture}[theorem]{Conjecture}
\crefname{conjecture}{Conjecture}{Conjectures}

%-----------------------------

\theoremstyle{definition}
\newtheorem{definition}[theorem]{Definition}
\crefname{definition}{Definition}{Definitions}

\newtheoremstyle{break}
  {\topsep}{\topsep}%
  {\itshape}{}%
  {\bfseries}{}%
  {\newline}{}%
\theoremstyle{break}
\newtheorem{breakproposition}{Proposition}
\newtheorem{breaklemma}[theorem]{Lemma}
\newtheorem{breaktheorem}[theorem]{Theorem}

%-----------------------------

\makeatletter
\newcommand{\proofpart}[2]{%
  \par
  \addvspace{\medskipamount}%
  \noindent\emph{Part #1: #2}\par\nobreak
  \addvspace{\smallskipamount}%
  \@afterheading
}
\makeatother
%-----------------------------


\theoremstyle{remark}
\newtheorem{remark}[theorem]{Remark}
\crefname{remark}{Remark}{Remarks}

\newtheorem{notation}[theorem]{Notation}
\crefname{notation}{Notation}{Notations}

\newtheorem{example}[theorem]{Example}
\crefname{example}{Example}{Examples}

%-----------------------------

\crefname{section}{Section}{Sections}
\crefname{subsection}{Subsection}{Subsections}

%-----------------------------

\newcommand{\pad}[2][2]{\hspace{#1mm} #2 \hspace{#1mm}}

\DeclareMathOperator{\A}{A}
\DeclareMathOperator{\B}{B}
\DeclareMathOperator{\C}{C}
\DeclareMathOperator{\dc}{d}
\DeclareMathOperator{\D}{D}
\DeclareMathOperator{\E}{E}
\DeclareMathOperator{\F}{F}
\DeclareMathOperator{\G}{G}
\DeclareMathOperator{\I}{I}
\DeclareMathOperator{\bI}{\mathbb{I}}
\DeclareMathOperator{\M}{M}
\DeclareMathOperator{\N}{N}
\DeclareMathOperator{\sCF}{\mathcal{CF}}
\DeclareMathOperator{\R}{R}
\DeclareMathOperator{\oS}{S}
\DeclareMathOperator{\SF}{SF}
\DeclareMathOperator{\GV}{GV}
\DeclareMathOperator{\T}{T}
\DeclareMathOperator{\sU}{\mathcal{U}}
\newcommand{\bN}{\mathbb{N}}
\newcommand{\sC}{\mathcal{C}}
\DeclareMathOperator{\sE}{\mathcal{E}}

\DeclareMathOperator{\support}{s}

\DeclareMathOperator{\dom}{dom}

\DeclareMathOperator{\MBF}{MBF}
\DeclareMathOperator{\BF}{BF}
\DeclareMathOperator{\NP}{NP}
\DeclareMathOperator{\coNP}{coNP}
\DeclareMathOperator{\QBF}{QBF}
\DeclareMathOperator{\QSAT}{QSAT}
\DeclareMathOperator{\QTAUT}{QTAUT}
\DeclareMathOperator{\TAUT}{TAUT}
\DeclareMathOperator{\COFBF}{COFBF}
\DeclareMathOperator{\COFENT}{COFENT}
\DeclareMathOperator{\TQBF}{TQBF}
\DeclareMathOperator{\SAT}{SAT}
\DeclareMathOperator{\UNSAT}{UNSAT}

\DeclareMathOperator{\TFPT}{TFPT}
\DeclareMathOperator{\EFPT}{EFPT}

\DeclareMathOperator{\PSPACE}{PSPACE}

\DeclareMathOperator{\id}{id}
\DeclareMathOperator{\true}{true}
\DeclareMathOperator{\Hom}{Hom}
\DeclareMathOperator{\HomC}{\Hom_{\sC}}
\DeclareMathOperator{\lvs}{lvs}
\DeclareMathOperator{\Var}{Var}

\newcommand{\concat}{\mathbin{;}}

\DeclareMathOperator{\Set}{Set}
\DeclareMathOperator{\necessarily}{\square}

\DeclareMathOperator{\arity}{arity}
\DeclareMathOperator{\qbf}{qbf}

\DeclareMathOperator{\rank}{rank}

\DeclareMathOperator{\apply}{a}

\DeclareMathOperator{\img}{img}

% Cubical Operators
\DeclareMathOperator{\is-Contr}{is-Contr}
\DeclareMathOperator{\fst}{fst}
\DeclareMathOperator{\snd}{snd}


\DeclareMathOperator{\outb}{out_{\psi}}
\DeclareMathOperator{\outp}{out_{[\alpha]}}
\DeclareMathOperator{\inp}{in_{[\alpha]}}
\DeclareMathOperator{\inb}{in_{\psi}}
\DeclareMathOperator{\case}{case}

\DeclareMathOperator{\shr}{shr}
\DeclareMathOperator{\flt}{flt}
\DeclareMathOperator{\br}{br}


\DeclareMathOperator{\cof}{\mathbf{cof}}

\DeclareMathOperator{\eM}{\mathbf{M}^*}
\DeclareMathOperator{\eN}{\mathbf{N}^*}
\DeclareMathOperator{\ee}{\mathbf{e}^*}
\DeclareMathOperator{\ef}{\mathbf{f}^*}
\DeclareMathOperator{\et}{\mathbf{t}^*}
\DeclareMathOperator{\eu}{\mathbf{u}^*}
\DeclareMathOperator{\bPsi}{\mathbf{\Psi}}



% Topology
\DeclareMathOperator{\sB}{\mathcal{B}}
\DeclareMathOperator{\sT}{\mathcal{T}}
\DeclareMathOperator{\cC}{\mathcal{C}}



\begin{document}
\maketitle
\tableofcontents

\section{Topological Spaces}

\begin{definition}[Lee]
  A \textit{toplogy} on a set $X$ is a collection $\mathcal{T}$ of subsets of $X$, called \textit{open sets}, satisfying the following:

  \begin{enumerate}
    \item $X \in \mathcal{T}$ and $\emptyset \in \mathcal{T}$

    \item If $U_{1}, \cdots, U_{n} \in \mathcal{T}$, then $U_{1} \cap \cdots \cap U_{n} \in \mathcal{T}$

    \item If $\{ U_{\alpha}\}_{\alpha \in A}$ is a collection of elements of $\mathcal{T}$, then $\cup_{\alpha \in A} U_{\alpha}$ is in $\mathcal{T}$.
  \end{enumerate}
\end{definition}

\begin{definition}[Lee]
  A \textit{topological space} is a pair $(X, \mathcal{T})$ consisting of a set $X$ and a topology $\mathcal{T}$ on $X$.
\end{definition}

\begin{remark}[Lee]
  A neighborhood of $q \in X$ is an open set containing $q$.
\end{remark}

\begin{lemma}[Exercise 2.1 (Munkres)]
 Let $(X, \sT)$ be a topological space; let $A$ be a subset of $X$. Suppose that for each $x \in A$ there is an open set $U$ containing $x$ such that $U \subset A$. Then, $A$ is open in $X$.
\end{lemma}
\begin{proof}
 Suppose that $A \subset X$ and that for all $x \in A$ there is an open set $U$ containing $x$ such that $U \subset A$. By definition, $\cup_{x \in A} U_{x} \in \sT$ because each $U_{x} \in \sT$. Then, $\cup_{x \in A} U_{x} = A$ so $A \in \sT$.
\end{proof}


\begin{lemma}[Exercise 2.3 (Munkres)]
  Is the collection
  \[\sT_{\inf} = \{U | X - U \text{ is infinite or empty or all of X}\}\]
  a topology on $X$?
\end{lemma}
\begin{proof}
  \begin{enumerate}
    \item[]
    \item If the condition is all of $X$, then $U = \emptyset$, so $\emptyset \in \sT$. If the condition is empty, then $U = X$, so $X \in \sT$.

    \item Let $U_{1}, \cdots, U_{n} \in \sT$. For $U = \cap_{i} U_{i}$, we have

          \[X - U =  X - \cap_{i} U_{i} = \cup_{i} (X - U_{i}) \].

          If any of the $X - U_{i}$ is infinite, all of X, or empty, the union meets the corresponding condition.

    \item Let $U_{1}, U_{2}, \cdots \in \sT$. For $U = \cup_{i} U_{i}$, we have

          \[X - U = X - \cup_{i} U_{i} = \cap_{i} ( X - U_{i} )\].

          If all of the $X - U_{i}$ are infinite, the intersection does not necessarily meet any of the conditions. Thus, $\sT_{\inf}$ is not a topology on $X$.
  \end{enumerate}
\end{proof}

\begin{definition}[Munkres]
  A topology $\sT$ is finer than $\sT'$ if $\sT' \subseteq \sT$ and, analogously, $\sT'$ is coarser.
\end{definition}

\subsection{Bases}

\begin{definition}[Munkres]
  If $X$ is a set, a \textit{basis} for a topology on $X$ is a collection $\mathcal{B}$ of subsets of $X$ such that
  \begin{enumerate}
    \item For each $x \in X$, there is at least one basis element $B$ containing $x$.
    \item If $x$ belongs to the intersection of two basis elements $B_{1}$ and $B_{2}$, then there is a basis element $B_{3}$ containing such that $B_{3} \subset B_{1} \cap B_{2}$.
  \end{enumerate}
\end{definition}

\begin{remark}
  If $\sB$ satisfies both of the above conditions, then we can define the topology $\mathcal{T}$ generated by $\sB$ as follows: A subset $U$ of $X$ is said to be open in $X$ if for each $x \in U$, there is a basis element $B \in \sB$ such that $x \in B$ and $B \subset U$. Each basis element itself is an element of $\mathcal{T}$.
\end{remark}

exercises + lemmas

\begin{lemma}[Munkres]
 Let $X$ be a topological space. Suppose that $\cC$ is a collection of open sets of $X$ such that for each open set $U$ and $x \in U$, there is $C \in \cC$ such that $x \in C \subset U$. Then $\sC$ is a basis for the topology of $X$.
\end{lemma}
\begin{proof}

  It suffices to show (1) that $\cC$ is a basis and (2) the topology generated by $\cC$ is the same as the topology of $X$.

  \proofpart{1}{$\cC$ is a basis}

  For the first condition, we must check that for each $x \in X$, there is a $C \in \cC$ containing $x$. Let $x \in X$. Because $X$ is an open set of $X$, we have, by supposition, that there is $C \in \cC$ such that $x \in C \subset X$.

  For the second condition, we must check that if $x \in C_{1} \cap C_{2}$, then there is $C_{3} \subset C_{1} \cap C_{2}$. Let $x \in C_{1} \cap C_{2}$. Because $C_{1}$ and $C_{2}$ are open, so is $C_{1} \cap C_{2}$. By supposition, there is $C_{3} \in \cC$ such that $x \in C_{3} \subset U$.


  \proofpart{2}{$\sC$ generates the topology of $X$}

  Let $\sT$ be the collection of open sets of $X$ and let $\sT'$ be the topology generated by $\cC$. Suppose that $U$ belongs to $\sT$.
\end{proof}

\subsection{Product Topology}

\begin{definition}[Lee]
  Suppose $X_{1}, \cdots, X_{n}$ are topological spaces. Then let $\sB = \{ U_{1} \times \cdots \times U_{n} : U_{i} \text{ is open in } X_{i}, i = 1, \cdots, n\}$. The topology generated by $\sB$ is the product topology.
\end{definition}

\begin{remark}
  We can say that the product topology $X \times Y$ is the topology that has as its basis $\sB$ the collection of all sets of the form $U \times V$ where $U$ is open in $X$ and $V$ is open in $Y$.

  Then, $\sB$ meets the first condition of a basis because $X \times Y$ is itself a basis element. For the second condition, consider $(U_{1} \times V_{1}) \cap (U_{2} \times V_{2}) = (U_{1} \cap U_{2}) \times (V_{1} \cap V_{2})$. Because the binary intersections are in $X$ and $Y$ (open), we have that the intersection is a basis element. We can generalize this to $n$-ary products.
\end{remark}

\begin{definition}
  Subspace topology ...

  ch3
\end{definition}

\begin{definition}
  Closed set and limit point ...
\end{definition}

\begin{definition}
  Continuous function ...
\end{definition}

\begin{definition}
  Metric space ....
\end{definition}

\begin{definition}
  Quotient topology ...
\end{definition}

\begin{definition}
  Connected space
\end{definition}

\begin{definition}
  Component and path component ...
\end{definition}

\begin{definition}
  Compact space ...
\end{definition}

\begin{definition}
  hausdorff space ...
\end{definition}

\begin{definition}
The separation axioms, Urysohn's lemma and the Tietze extension theorem
\end{definition}

\begin{definition}
  homotopy and the fundamental group ...
  ch7
\end{definition}

\begin{definition}
  group theory ...

  ch9
\end{definition}

\end{document}

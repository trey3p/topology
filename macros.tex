%===========================================================
\usepackage{amsmath,amsthm,amssymb,stmaryrd}
\usepackage[alphabetic]{amsrefs}
\usepackage{hyperref, cleveref}
\usepackage[inline]{enumitem}
\usepackage{lipsum}

\theoremstyle{plain}
\newtheorem{theorem}{Theorem}[section]
\crefname{theorem}{Theorem}{Theorems}

\newtheorem{lemma}[theorem]{Lemma}
\crefname{lemma}{Lemma}{Lemmas}

\newtheorem{proposition}[theorem]{Proposition}
\crefname{proposition}{Proposition}{Propositions}

\newtheorem{corollary}[theorem]{Corollary}
\crefname{corollary}{Corollary}{Corollaries}

\newtheorem{conjecture}[theorem]{Conjecture}
\crefname{conjecture}{Conjecture}{Conjectures}

%-----------------------------

\theoremstyle{definition}
\newtheorem{definition}[theorem]{Definition}
\crefname{definition}{Definition}{Definitions}

\newtheoremstyle{break}
  {\topsep}{\topsep}%
  {\itshape}{}%
  {\bfseries}{}%
  {\newline}{}%
\theoremstyle{break}
\newtheorem{breakproposition}{Proposition}
\newtheorem{breaklemma}[theorem]{Lemma}
\newtheorem{breaktheorem}[theorem]{Theorem}

%-----------------------------

\makeatletter
\newcommand{\proofpart}[2]{%
  \par
  \addvspace{\medskipamount}%
  \noindent\emph{Part #1: #2}\par\nobreak
  \addvspace{\smallskipamount}%
  \@afterheading
}
\makeatother
%-----------------------------


\theoremstyle{remark}
\newtheorem{remark}[theorem]{Remark}
\crefname{remark}{Remark}{Remarks}

\newtheorem{notation}[theorem]{Notation}
\crefname{notation}{Notation}{Notations}

\newtheorem{example}[theorem]{Example}
\crefname{example}{Example}{Examples}

%-----------------------------

\crefname{section}{Section}{Sections}
\crefname{subsection}{Subsection}{Subsections}

%-----------------------------

\newcommand{\pad}[2][2]{\hspace{#1mm} #2 \hspace{#1mm}}

\DeclareMathOperator{\A}{A}
\DeclareMathOperator{\B}{B}
\DeclareMathOperator{\C}{C}
\DeclareMathOperator{\dc}{d}
\DeclareMathOperator{\D}{D}
\DeclareMathOperator{\E}{E}
\DeclareMathOperator{\F}{F}
\DeclareMathOperator{\G}{G}
\DeclareMathOperator{\I}{I}
\DeclareMathOperator{\bI}{\mathbb{I}}
\DeclareMathOperator{\M}{M}
\DeclareMathOperator{\N}{N}
\DeclareMathOperator{\sCF}{\mathcal{CF}}
\DeclareMathOperator{\R}{R}
\DeclareMathOperator{\oS}{S}
\DeclareMathOperator{\SF}{SF}
\DeclareMathOperator{\GV}{GV}
\DeclareMathOperator{\T}{T}
\DeclareMathOperator{\sU}{\mathcal{U}}
\newcommand{\bN}{\mathbb{N}}
\newcommand{\sC}{\mathcal{C}}
\DeclareMathOperator{\sE}{\mathcal{E}}

\DeclareMathOperator{\support}{s}

\DeclareMathOperator{\dom}{dom}

\DeclareMathOperator{\MBF}{MBF}
\DeclareMathOperator{\BF}{BF}
\DeclareMathOperator{\NP}{NP}
\DeclareMathOperator{\coNP}{coNP}
\DeclareMathOperator{\QBF}{QBF}
\DeclareMathOperator{\QSAT}{QSAT}
\DeclareMathOperator{\QTAUT}{QTAUT}
\DeclareMathOperator{\TAUT}{TAUT}
\DeclareMathOperator{\COFBF}{COFBF}
\DeclareMathOperator{\COFENT}{COFENT}
\DeclareMathOperator{\TQBF}{TQBF}
\DeclareMathOperator{\SAT}{SAT}
\DeclareMathOperator{\UNSAT}{UNSAT}

\DeclareMathOperator{\TFPT}{TFPT}
\DeclareMathOperator{\EFPT}{EFPT}

\DeclareMathOperator{\PSPACE}{PSPACE}

\DeclareMathOperator{\id}{id}
\DeclareMathOperator{\true}{true}
\DeclareMathOperator{\Hom}{Hom}
\DeclareMathOperator{\HomC}{\Hom_{\sC}}
\DeclareMathOperator{\lvs}{lvs}
\DeclareMathOperator{\Var}{Var}

\newcommand{\concat}{\mathbin{;}}

\DeclareMathOperator{\Set}{Set}
\DeclareMathOperator{\necessarily}{\square}

\DeclareMathOperator{\arity}{arity}
\DeclareMathOperator{\qbf}{qbf}

\DeclareMathOperator{\rank}{rank}

\DeclareMathOperator{\apply}{a}

\DeclareMathOperator{\img}{img}

% Cubical Operators
\DeclareMathOperator{\is-Contr}{is-Contr}
\DeclareMathOperator{\fst}{fst}
\DeclareMathOperator{\snd}{snd}


\DeclareMathOperator{\outb}{out_{\psi}}
\DeclareMathOperator{\outp}{out_{[\alpha]}}
\DeclareMathOperator{\inp}{in_{[\alpha]}}
\DeclareMathOperator{\inb}{in_{\psi}}
\DeclareMathOperator{\case}{case}

\DeclareMathOperator{\shr}{shr}
\DeclareMathOperator{\flt}{flt}
\DeclareMathOperator{\br}{br}


\DeclareMathOperator{\cof}{\mathbf{cof}}

\DeclareMathOperator{\eM}{\mathbf{M}^*}
\DeclareMathOperator{\eN}{\mathbf{N}^*}
\DeclareMathOperator{\ee}{\mathbf{e}^*}
\DeclareMathOperator{\ef}{\mathbf{f}^*}
\DeclareMathOperator{\et}{\mathbf{t}^*}
\DeclareMathOperator{\eu}{\mathbf{u}^*}
\DeclareMathOperator{\bPsi}{\mathbf{\Psi}}



% Topology
\DeclareMathOperator{\sB}{\mathcal{B}}
\DeclareMathOperator{\sT}{\mathcal{T}}
\DeclareMathOperator{\cC}{\mathcal{C}}
\DeclareMathOperator{\sD}{\mathcal{D}}
